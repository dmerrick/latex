\documentclass[12pt]{amsart}

\textwidth=1.25\textwidth
\calclayout


\begin{document}

\thispagestyle{empty}

\begin{center}
{\bf Math 331  --- Homework 9 \\
Due:  Noon, Tuesday, May 4}
\end{center}

\bigskip

\noindent
Dana Merrick \\
\today

\bigskip


\begin{enumerate}

\setlength{\itemsep}{6pt}

\item Given a function $f : [a,b] \rightarrow \mathbb{R}$, suppose $f$ is continuous everywhere on $[a,b]$ and differentiable everywhere on $(a,b)$.  Prove that if $f'(x)>0$ on $(a,b)$, then $f$ is increasing and
  if $f'(x)<0$, then $f$ is decreasing.

\begin{proof}
Suppose that $f'(x) > 0$ for all $x\in (a,b)$. Let $x_1,\, x_2 \in [a,b]$. Without loss of generality, we can say $x_1 < x_2$. By the Mean Value Theorem, there exists a value $c$, $x_1 < c < x_2$, such that:
%
\[ f'(c) = \frac {f(x_2) - f(x_1)}{x_2 - x_1}. \]
%
Since $f'(c) > 0$ and $x_1 < x_2$, we know that $f(x_2) - f(x_1) > 0$. Therefore $f(x_1) < f(x_2)$, which implies that $f$ is increasing on $[a,b]$.

\bigskip

Similarly, suppose that $f'(x) < 0$ for all $x\in (a,b)$. Let $x_1,\, x_2 \in [a,b]$. Without loss of generality, we can say $x_1 < x_2$. By the Mean Value Theorem, there exists a value $c$, $x_1 < c < x_2$, such that:
%
\[ f'(c) = \frac {f(x_2) - f(x_1)}{x_2 - x_1}. \]
%
Since $f'(c) < 0$ and $x_1 < x_2$, we know that $f(x_2) - f(x_1) < 0$. Therefore $f(x_1) > f(x_2)$, which implies that $f$ is decreasing on $[a,b]$.
\end{proof}

\item Prove the first derivative test:  Given a function $f : [a,b] \rightarrow \mathbb{R}$, suppose $f$ is continuous everywhere on $[a,b]$ and differentiable everywhere on $(a,c)$ and $(c,b)$ for some $c\in (a,b)$.    If $f$ has a critical
  point at $c\in (a,b)$, (i.e., either $f'(c)$ does not exist or $f'(c)=0$) and if $f'(x)>0$ for $x<c$ and $f'(x)<0$ for $x>c$,  then has a local maximum at $c$.  Formulate and prove the corresponding result for a local minimum.

\begin{proof}
Claim: If $f$ has a critical  point at $c\in (a,b)$, (i.e., either $f'(c)$ does not exist or $f'(c)=0$) and if $f'(x)<0$ for $x<c$ and $f'(x)>0$ for $x>c$, then has a local minimum at $c$.

Suppose $f$ has a critical  point at $c\in (a,b)$ and $f'(x)<0$ for $x<c$ and $f'(x)>0$ for $x>c$. Then there exists $a,\,b \in [a,b]$ such that $f'(x) < 0$ for all $x\in (a,c)$ and $f'(x) > 0$ for all $x\in (c,b)$. By Problem 1, $f$ is decreasing on $[a,c]$ and increasing on $[c,b]$. Therefore, $f(c)$ is a minimum of $f$ on $(a,b)$.
\end{proof}

\item Given the function $f : [1,3] \rightarrow \mathbb{R}$,
  $f(x)=7-2x$, use the definition to prove that $f$ is integrable and determine the value of 
%
\[ \int_1^3  f(x) \,dx. \]

\begin{proof}
Let $P_n = \{ 1 + \tfrac {3-1} n i \}_{i=0}^n$ be a regular partition of $[1,2]$. Fix $i$ and consider $f$ on $[ x_{i-1}, x_i ]$. Since $f$ is decreasing, we have that $m_i (f,P_n) = f(x_i)$ and $M_i(f,P_n) = f(x_{i-1})$. So:
%
\[ f(x_i) = f\left(1 + \frac{2i}{n}\right) = 7 - 2\left(1 + \frac{2i}{n}\right) = 5 - \frac{4i}{n}, \]
%
and,
%
\[ f(x_{i-1}) = f\left(1 + \frac{2(i-1)}{n}\right) = 7 - 2\left(1 + \frac{2(i-1)}{n}\right) = 5 - \frac{4(i-1)}{n}. \]
%
Therefore,
%
\begin{align*}
L(f,P_n) &= \sum_{i=1}^n m_i(f,P_n) \Delta x = \sum_{i=1}^n \left(5 - \frac{4i}{n} \right) \frac 1 n \\
&= \frac 5 n \sum_{i=1}^n 1 - \frac 4 {n^2} \sum_{i=1}^n i = \frac 5 n n - \frac 4 {n^2} \cdot \frac {n(n+1)} 2 \\
&= 5 - \frac {2(n+1)}{n} = \frac {3n-2} n.
\end{align*}
%
Similarly, $U(f,P_n) = \tfrac {3n+2} n$. So we have that,
%
\[ \inf\{ U(f,P_n) : n\in\mathbb N \} = 3, \]
%
and,
%
\[ \ge \inf\{ U(f,P) : P \textrm{ partition of }[1,3] \} = \overline\int_a^b f(x)dx. \]
%
Similarly,
%
\[ \sup\{ L(f,P) : P \textrm{ partition of }[1,3] \} \ge \sup\{ L(f,P_n) : n\in\mathbb N \} = 3. \]
%
In other words,
%
\[ 3 \le \underline\int_1^3 f(x) dx \le \overline\int_1^3 f(x) dx \le 3, \]
%
and therefore the upper and lower values are equal.
\end{proof}

\item Use the definition to prove that the function $f : [a,b]
  \rightarrow \mathbb{R}$, $f(x)=x$, is integrable and that
%
\[ \int_a^b f(x)\,dx = \frac{b^2-a^2}{2}. \]

\begin{proof}
Let $P_n = \{ a + \tfrac {b-a} n i \}_{i=0}^n$ be a regular partition of $[a,b]$. Fix $i$ and consider $f$ on $[ x_{i-1}, x_i ]$. Since $f$ is increasing, we have that $m_i (f,P_n) = f(x_{i-1})$ and $M_i(f,P_n) = f(x_i)$. So:
%
\[ f(x_i) = f\left(a + \frac{(b-a)i}{n}\right) = \left(a + \frac{(b-a)i}{n}\right) = \frac {-ai + an + bi} n , \]
%
and,
%
\begin{align*}
f(x_{i-1}) &= f\left(a + \frac{(b-a)(i-1)}{n}\right) = \left(a + \frac{(b-a)(i-1)}{n}\right) \\
&= \frac {-ai + an + a + bi - b} n.
\end{align*}
%
Therefore,
%
\begin{align*}
U(f,P_n) &= \sum_{i=1}^n M_i(f,P_n) \Delta x = \sum_{i=1}^n \frac {-ai + an + bi} n \cdot \frac 1 n \\
&= \frac {an - a + bn + b} {2n}.
\end{align*}
%
Similarly,
%
\[ L(f,P_n) = \frac{an+a+bn-b}{2n}. \]
%
So $\inf\{ U(f,P_n) : n \in\mathbb N \} = b$ and $\sup\{ L(f,P_n) : n \in\mathbb N \} = a$.
\end{proof}

\item Prove that if $f : [a,b] \rightarrow \mathbb{R}$ is increasing, then $f$ is integrable.
%
%  Hint: This is similar to the proof that continuous functions are
%  integrable.  In that proof we chose the partition $P$ so that
%  $M_i(f,P)-m_i(f,P)$ was small and then used the fact that
%%
%\[ \sum_{i=1}^n \Delta x_i = b -a, \]
%%
%that is, this is a telescoping sum.  To prove this, again use the
%Darboux criterion and use the fact that increasing function on an
%interval $[c,d]$ takes on its infimum at $c$ and its supremum at $d$.
%Fix a regular partition $P$ and show that
%%
%\[ \sum_{i=1}^n M_i(f,P)-m_i(f,P) \]
%%
%is a telescoping sum.   Choose the partition so that the width
%$\Delta x_i$ is small---make it small enough to get the
%``$\epsilon$'' you need to apply the Darboux criterion.

\begin{proof}
Fix $P_n$, a regular partition. Then each $x_i - x_{i-1} = \tfrac{b-a}{n} = \Delta x$. Note that if $f$ is increasing on an interval $[c,d]$, then it takes on its infimum at $c$ and its supremum at $d$. Therefore,
%
\[ U(f, P_n) - L(f, P_n) = \sum_{i=1}^n f(x_i) \Delta x - \sum_{i=1}^n f(x_{i-1}) \Delta x, \]
%
and after combining the two sums and replacing $\Delta x$ with $\tfrac{b-a}{n}$ we get,
%
\begin{align*}
&= \frac{b-a}{n} \sum_{i=1}^n f(x_i) - f(x_{i-1}) \\
&= f(x_1) - f(x_0) + f(x_2) - f(x_1) + \ldots + f(x_n) - f(x_{n-1}),
\end{align*}
%
which is the same as saying,
%
\[ = \frac{b-a}{n} \left( f(x_n) - f(x_0) \right) = \frac{b-a}{n} \left( f(b) - f(a) \right). \]

Fix $\epsilon > 0$. By the Archimedean principle, we can choose an $n$ for $P_n$ sufficiently large such that:
\[ U(f, P_n) - L(f, P_n) = \frac{b-a}{n} \left( f(b) - f(a) \right) < \epsilon. \]
Hence, $f$ is integrable.
\end{proof}

%\item (Extra Credit)  Prove that the function $f : [a,b] \rightarrow \mathbb{R}$, $f(x)=x^2$, is integrable and that 
%%
%\[ \int_a^b f(x)\,dx = \frac{b^3-a^3}{3}. \]
%
%
%\item (Extra Credit) Prove that if the function $f : [a,b] \rightarrow
%  \mathbb{R}$ is integrable, then so is $|f|$, and 
%%
%\[ \left| \int_a^b f(x)\,dx \right| \leq \int_a^b |f(x)|\,dx. \]
%%
%
%Hint:  Use the Darboux criterion.
%
%\item (Extra Credit) Prove that if $f,\,g : [a,b] \rightarrow
%  \mathbb{R}$ are integrable, and $c\in \mathbb{R}$, then $cf$ and
%  $f+g$ are integrable and
%%
%\begin{gather*}
%\int_a^b cf(x)\,dx = c \int_a^b f(x)\,dx \\
%\int_a^b \big( f(x)+g(x)\big) \,dx = \int_a^b f(x)\,dx + \int_a^b
%g(x)\,dx. 
%\end{gather*}
%
%Hint:  For $cf$ use the Darboux criterion.  Do you need to treat the
%case $c=0$ separately?  For $f+g$, use the Darboux criterion and a
%common refinement of two partitions.

\end{enumerate}
\end{document}
