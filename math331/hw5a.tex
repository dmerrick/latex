\documentclass[12pt]{amsart}

\textwidth=1.25\textwidth
\calclayout


\begin{document}

\thispagestyle{empty}

\begin{center}
{\bf Math 331  --- Homework 5 \\
Due:  Friday, March 12}
\end{center}

\bigskip

\noindent
Dana Merrick \\
\today

\bigskip

\begin{enumerate}

\setlength{\itemsep}{6pt}

\addtocounter{enumi}{1}

\item Prove that $\{\sqrt{1+1/n} \}_{n=1}^\infty$ converges to $1$ as $n\rightarrow \infty$. 

Hint:  Use the identity $(\sqrt{x}-\sqrt{y})(\sqrt{x}+\sqrt{y})=x-y$.
  
\begin{proof}
Fix $\epsilon>0$. We need to find $N>0$ such that if $n\ge N$,
\[ \left| \sqrt{1+\frac 1 n} - 1\right| < \epsilon.\]
We can disregard the absolute value signs because $\sqrt{1+\frac 1 n} \ge 1$ for all $n$. Since $(\sqrt{x}-\sqrt{y})(\sqrt{x}+\sqrt{y})=x-y$, we can multiply both sides by $\frac {\sqrt{1+\frac 1 n} + 1}{\sqrt{1+\frac 1 n} + 1}$ to get:
\begin{align*}
\left(\sqrt{1+\frac 1 n} - 1\right)\cdot \left(\frac {\sqrt{1+\frac 1 n} + 1}{\sqrt{1+\frac 1 n} + 1} \right) &= \frac {(1+\frac 1 n) - 1}{\sqrt{1+\frac 1 n} + 1} \\
&= \frac {\frac 1 n}{\sqrt{1+\frac 1 n} + 1}.
\end{align*}
We need to show that: 
\[ \frac {\frac 1 n}{\sqrt{1+\frac 1 n} + 1} < \epsilon. \]
Since $\sqrt{1+\frac 1 n} + 1 \ge  2$, we have that:
\[\frac {\frac 1 n}{\sqrt{1+\frac 1 n} + 1} < \frac {\left(\frac 1 n\right)}{2} = \frac 1 {2n} < \epsilon.
\]
By the Archimedean property, there exists $N\in\mathbb N$ such that $N > \frac 1 {2\epsilon}$. Therefore, for all $n\ge N$,
\[ 
\frac 1 n \le \frac 1 N \le \frac 1 {2\epsilon}.
\]
Hence, $\frac 1 {2n} < \epsilon$.
\end{proof}

\item Prove that given the sequence $\{a_n\}_{n=1}^\infty$, if
  $a_n\geq 0$ and $a_n\rightarrow L$ as $n\rightarrow \infty$, then
  $\sqrt{a_n}\rightarrow \sqrt{L}$ as $n\rightarrow \infty$. 
  
\begin{proof}
Fix $\epsilon>0$. We need to find $N>0$ such that if $n\ge N$,
\[ \left| \sqrt{a_n} - \sqrt{L}\right| < \epsilon.\]
We will proceed by cases.

Case 1: $L=0$.

 If $L=0$, then $\sqrt L = 0$, so:
\[ \left| \sqrt{a_n} - \sqrt{L}\right| = \left| \sqrt{a_n} - 0\right| = |\sqrt {a_n}| < \epsilon.\]
But we have that $a_n \ge 0$ so $\sqrt {a_n} \ge 0$ for all $n$. Since $a_n\to 0$ as $n\to \infty$, we can choose a sufficiently large $n$ such that $a_n < \epsilon$.

Case 2: $L>0$.

If $L>0$, then we want to show that:
 \[ \left| \sqrt{a_n} - \sqrt{L}\right| < \epsilon.\]
Since both $\sqrt{a_n}$ and $\sqrt L$ are positive, we can disregard the absolute value signs. Multiplying by $\frac {\sqrt{a_n} + \sqrt{L}}{\sqrt{a_n} + \sqrt{L}}$, we get the following:
\begin{align*}
\sqrt{a_n} - \sqrt{L}&= \left(\sqrt{a_n} - \sqrt{L}\right)\cdot \left(\frac {\sqrt{a_n} + \sqrt{L}}{\sqrt{a_n} + \sqrt{L}}\right) \\
&= \frac {a_n - L} {\sqrt{a_n} + \sqrt{L}}
\\&< \epsilon
\end{align*}
We need to show that we can make $\frac {a_n - L} {\sqrt{a_n} + \sqrt{L}}$ smaller than $\epsilon$. Since we know that $a_n\rightarrow L$ as $n\rightarrow \infty$, for all $\delta > 0$ there exists an $M>0$ such that for all $n\ge M$, $|a_n - L| < \delta$.
\end{proof}
\end{enumerate}


\end{document}
