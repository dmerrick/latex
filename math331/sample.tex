% Sample HW assignment, demonstrating some bare bones of LaTeX
%
% Version of January 20, 2010

\documentclass[12pt]{amsart}

% Always use the 12pt option.  The amsart option invokes a number of
% additional style files created by the American Mathematical Society

% The following command allows me to insert a comment box into the
% document.  You will probably never need this when you actually do homework.

\newcommand{\comment}[1]{\vskip.3cm
\fbox{%
\parbox{0.93\linewidth}{\footnotesize #1}}
\vskip.3cm}

\begin{document}

% The commend \noindent keeps LaTeX from indenting my name as the
% first line of this "paragraph," by the default amount.

\noindent
David Cruz-Uribe, SFO \\
Sample Homework  \\
January 20, 2010

\bigskip

\begin{enumerate}

\item Prove that if $a,\,b$ are positive real numbers, then
%
\[\sqrt{ab}\leq \frac{a+b}{2}. \]
%

\begin{proof}
  Since $a$ and $b$ are positive, $\sqrt{a}$, $\sqrt{b}$ and
  $\sqrt{ab}$ exist; furthermore, $\sqrt{ab}=\sqrt{a}\sqrt{b}$ and
  $(\sqrt{a})^2=a$ and the same holds for $b$.  Since the square of
  any real number is non-negative, $(\sqrt{a}-\sqrt{b})^2\geq 0$.  If
  we expand the left-hand side we get
%
\[ (\sqrt{a})^2 -2\sqrt{a}\sqrt{b} + (\sqrt{b})^2 \geq 0.  \]
%
If we simplify and re-arrange terms we get
%
\[ a + b \geq 2\sqrt{ab}, \]
%
and dividing by $2$ we get the desired inequality.
\end{proof}

\comment{The first line of this proof may seem pedantic, and in more
  advanced arguments you may assume this properties without
  comment.  However, when proving elementary inequalities or anything
  else that works straight from the definitions or other basic
  properties, it is a good idea to make sure that it is clear what you
  are using.  In particular, this sentence makes it clear how I am
  using the hypothesis that $a,\,b$ are positive.   As part of a
  longer argument, it would be acceptable to omit this sentence and
  replace the line ``If we simplify and re-arrange terms...'' by
  ``Since $a,\,b$ are positive, we can simplify and re-arrange terms
  to get''.    The reader should never have to dig to find where you
  use your hypotheses:  make it clear!}

\item A game consists of a chessboard with dimensions $2^n \times 2^n$
  and an unlimited number of $L$-shaped tiles. Each $L$-shaped tile is
  congruent to the letter $L$ as formed by three squares on the
  chessboard, so each tile will precisely cover three such
  squares. Prove that, for any positive integer $n$, the $2^n \times
  2^n$ chessboard can be completely tiled (no tiles may overlap) with
  the exception of one lone arbitrarily chosen square.

\begin{proof}
We will prove this by induction.  When $n=1$, the board is $2\times
2$.  By rotating it, a single $L$-shaped tile can be placed to leave any
one of the four squares open.

Now suppose that for a fixed positive integer $n$, it is true that a
$2^n\times 2^n$ chessboard can be covered with $L$-shaped tiles,
leaving any arbitrarily chosen square open.    Now consider a
$2^{n+1}\times 2^{n+1}$ chessboard, and pick any square on it.  This
chessboard may be divided into four $2^n\times 2^n$ squares by
bisecting each side.  Number these squares, starting in the upper
right-hand corner and proceeding counter-clockwise, as $I$, $II$,
$III$ and $IV$.   By rotating the board we may assume that the square
designated above is in square $I$.  By our induction hypothesis, we
may cover square $I$ with $L$-shaped tiles leaving this one square
uncovered.     Again by our induction hypothesis, we can cover square
$II$ so that its lower right-hand corner is uncovered; we can cover
square $III$ so that its upper right-hand corner is uncovered; and
cover square $IV$ so that its upper left-hand corner is uncovered.
Finally, these three uncovered squares are adjacent and form an
$L$-shape, so we can cover it with one additional tile.  

Thus we have shown that we can cover a $2^{n+1}\times 2^{n+1}$ board
with $L$-shaped tiles.  Therefore, by induction, it is possible to do
this for all positive integers $n$.
\end{proof} 

\comment{In any proof that is not a direct proof, always tell your
  reader at some point how the proof will go.  In an induction proof,
  it is customary to use the same variable name as a variable
  representing all positive integers and the fixed value in the
  induction step.  Note that I wrote ``for a fixed positive integer
  $n$'' to signal that $n$ now had a definite value, and then at the
  end of the proof I wrote that the result holds for ``all positive
  integers $n$'' to signal I was referring to all possible values.
  If you prefer, you can use a different variable in the induction
  step.   It is also customary to conclude the proof by simply saying
  ``by induction'' rather than the more formal ``by the principle of
  mathematical induction.''   Similarly, the last sentence can be
  stated more formally.  This phrasing is as informal as it should
  get, however.}


\end{enumerate}

\end{document}
