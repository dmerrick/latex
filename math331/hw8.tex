\documentclass[12pt]{amsart}

\textwidth=1.25\textwidth
\calclayout


\begin{document}

\thispagestyle{empty}

\begin{center}
{\bf Math 331  --- Homework 8 \\
Due:  Monday, April 26}
\end{center}

\bigskip

\noindent
Dana Merrick \\
\today
\bigskip


\begin{enumerate}

\setlength{\itemsep}{6pt}

\item Given a function $f : D \rightarrow \mathbb{R}$, suppose that for some $x\in D$, and every sequence $\{x_n\}_{n=1}^\infty \subset D$ such that $x_n\rightarrow x$ as $n\rightarrow \infty$, $f(x_n)\rightarrow f(x)$ as $n\rightarrow \infty$.   Prove that $f$ is continuous at $x$.
%
%Hint:  Try proving the contrapositive.

\begin{proof}
We will proceed by contraposition. We want to show that if $f$ is discontinuous at $x$, then $f(x_n)\not\to f(x)$ as $n\to\infty$. If $f$ is discontinuous at $x$, then there exists an $\epsilon > 0$ such that for any $\delta >0$ there exists a $x_m\in D$ such that $|x_n -x_m| < \delta$ but $|f(x_n)-f(x_m)| \ge \epsilon$.

Fix $\delta >0$ and let $x_m\in D$ be such that $|x_n -x_m| < \delta$.
\end{proof}

\addtocounter{enumi}{1}
%\item Prove that if $f: [a,b] \rightarrow \mathbb{R}$ is continuous everywhere on $[a,b]$, then $f([a,b])$ is an interval.
%
%Hint:  Go back to the beginning of the semester and recall what it
%means to say $f([a,b])=[c,d]$.   There are two parts to the proof and
%it will require results proved in class.
%
%\begin{proof}
%Proof goes here.
%\end{proof}

\item Complete the proof of the extreme value theorem given in class by showing that there exists a point $x_{min}$ such that $f(x_{min})= y_{min}$.  
%
%Hint:  The hypotheses and notation are the same as in the proof from class, but in your write up, be sure to define all of these terms.

\begin{proof}
We have already shown that given $f:[a,b]\to\mathbb R$ where $f$ is continuous everywhere on $[a,b]$, the set of values $\{ f(x) : x\in [a,b] \}$ is bounded. By the completeness axiom, its supremum and infimum exist. Denote these values by $y_{max}$ and $y_{min}$. We have that $y_{min} \le f(x) \le y_{max}$ for all $x\in[a,b]$.

To finish the proof we need to find $x_{min},\,x_{max} \in [a,b]$ such that $f(x_{min}) = y_{min}$. Construct $\{ x_n \}_{n=1}^\infty$ as follows. For all $n\in\mathbb N$, there exists an $x_n\in [a,b]$ such that $y_{min} < f(x_n) < y_{min} + \tfrac 1 n$, $f(x_n) \ge y_{min}$ since $y_{min}$ is the infimum. If $x_n$ does not exist, then $f(x) \ge y_{min} + \tfrac 1 n$ for all $x\in [a,b]$. This contradicts the fact that $y_{min}$ is the greatest lower bound.

Since $y_{min} + \tfrac 1 n \to y_{min}$ as $n\to\infty$, by the squeeze lemma, $f(x_n)\to y_{min}$ as $n\to\infty$. This gives us a sequence $\{ x_n \}_{n=1}^\infty \subset [a,b]$. Since this sequence is bounded, it has a convergent subsequence by the Bolzano-Weierstrauss theorem. So there exists $\{ x_{n_k} \}_{k=1}^\infty$ and $x_{min}$ such that $x_{n_k} \to x_{min}$ as $k \to \infty$.

Claim: $x_{min} \in [a,b]$, since $a \le x_{n_k} \le b$ by a homework problem $a \le x_{min} \le b$. Therefore $f(x_{min})$ exists and by the lemma from class since $f$ is continuous at $x_{min}$, $f(x_{n_k}) \to y_{min}$ as $k\to\infty$ by the other in class lemma $f(x_{min}) = y_{min}$.
\end{proof}

\item Complete the proof of the intermediate value theorem by doing the case where $x_{max}<x_{min}$.  

\begin{proof}
Case $x_{max}<x_{min}$: Define $E=\{z\in [a,b] : f(z) > 0 \}$. Since $f(x_{max}) > 0$, E is nonempty, and since it's contained in $[x_{min}, x_{max}]$, E is bounded. By the completeness axiom, $\inf E$ exists and $\inf E \ge x_{min}$. Let $x=\inf E$. I claim $f(x)=0$.

Suppose to the contrary that $f(x) \not = 0$. If $f(x) > 0$, by continuity there exists a $\delta > 0$ such that if $|x-y|<\delta$, $|f(x)-f(y)|<\tfrac {|f(x)|} 2$. This implies $f(y)-f(x) < \tfrac {-f(x)} 2$ and so $f(y) < \tfrac {-f(x)} 2 < 0$. We can find points $y$ such that $x-\delta<y<x$ and $f(y)<0$. So $y \in E$, contradicting that $x=\inf E$.

If $f(x) < 0$, since $f(x_{max}) > 0$, $x<x_{max}$.
\end{proof}

\item Use the definition to prove that $f: \mathbb{R} \rightarrow \mathbb{R}$, $f(x)=x^{1/3}$, is differentiable at every $x\neq 0$.  

\begin{proof}
Fix $x\in\mathbb R\backslash \{0\}$. We need to show $\lim_{y\to x} \tfrac {f(x) - f(y)} {x-y} = \tfrac 1 {3 x^{2/3}}$. In other words:
%
\[ \lim_{y\to x} \frac {f(x) - f(y)} {x-y} = \lim_{y\to x} \frac {x^{1/3} - y^{1/3}} {x-y}. \]
%
Using the limit rules, we can write this as:
%
\[ \lim_{y\to x} \frac {x^{1/3} - y^{1/3}} {x-y} =  \lim_{y\to x} \frac 1 {x^{2/3} + x^{1/3}\cdot y^{1/3} + y^{2/3}}. \]
%
By the continuity of $\tfrac 1 {y^{1/3}}$ and $\tfrac 1 {y^{2/3}}$, we have:
%
\begin{align*}
 \lim_{y\to x} \frac 1 {x^{2/3} + x^{1/3}\cdot y^{1/3} + y^{2/3}} &=
\frac 1 {x^{2/3} + x^{1/3}\cdot x^{1/3} + x^{2/3}} \\
 &= \frac 1 {x^{2/3} + x^{2/3}+ x^{2/3}} \\
 &= \frac 1 {3 x^{2/3}}.
\end{align*}
%
\end{proof}

\item Prove that if $f : D \rightarrow \mathbb{R}$ is uniformly continuous, then given any Cauchy sequence $\{x_n\}\subset D$, the sequence $\{f(x_n)\}$ is also a Cauchy sequence.

Hint:  Use the definition of Cauchy sequence and the definition of uniform continuity.  Choose the value of ``$\epsilon$'' in the definition carefully!

\begin{proof}
Let $f : D \rightarrow \mathbb{R}$ be a uniformly continuous function, and $\{x_n\}\subset D$ be a Cauchy sequence. Fix $\epsilon > 0$. By the definition of a Cauchy sequence, there exists an $N > 0$ such that for all $n,m \ge N$, $|x_n - x_m| < \epsilon$. Additionally, by the definition of a uniformly continuous function, there exists a $\delta > 0$ such that for all $x_n,\,x_m\in D$ such that $|x_n-x_m|<\delta$, $|f(x_n)-f(x_m)|<\epsilon$.

We want to show that there exists an $N > 0$ such that for all $n,m \ge N$, $|f(x_n) - f(x_m)| < \epsilon$. But we already know from the definition of a uniformly continuous function that for all $x_n,\,x_m\in D$ such that $|x_n-x_m|<\delta$, $|f(x_n)-f(x_m)|<\epsilon$. Let $N=\min(n,m)$ where $|x_n-x_m|<\delta$. Then for all $n,m \ge N$, $|f(x_n) - f(x_m)| < \epsilon$, and thus $\{f(x_n)\}$ is a Cauchy sequence.
\end{proof}

%\item (Extra Credit)  Let $f : [a,b) \rightarrow \mathbb{R}$ be uniformly continuous.  Show that there exists a value $L$ such that if we define $f(b)=L$, then $f$ is continuous at $b$.  
%
%Hint:  Fix a Cauchy sequence in $[a,b)$ that converges to $b$ and use the previous problem to find what the limit should be equal to.  Then prove the function is continuous at $b$.  
%
%\begin{proof}
%Proof goes here.
%\end{proof}

\addtocounter{enumi}{1}

\item (Extra Credit) Construct a function $S : \mathbb{R}\rightarrow \mathbb{R}$, 
that is continuous only at $0$. 
%
%Hint:  start with the function $R$ defined in class that is not continuous anywhere;  can you modify the definition to make it continuous at $0$?

\begin{equation*}
f(x)= 
\begin{cases}
x  &\text{if $x$ is rational} \\
-x &\text{if $x$ is irrational}
\end{cases}
\end{equation*}

\end{enumerate} 

\end{document}
