\documentclass[12pt]{amsart}

\textwidth=1.25\textwidth
\calclayout


\begin{document}

\thispagestyle{empty}

\begin{center}
{\bf Math 331  --- Homework 6 \\
Due:  Friday, March 19}
\end{center}

\bigskip

\noindent
Dana Merrick \\
\today

\bigskip


\begin{enumerate}

\setlength{\itemsep}{6pt}


\item Prove that if $\{a_n\}_{n=1}^\infty$ is a Cauchy
  sequence, then it is bounded.

Hint:  follow the proof that convergent sequences are bounded.

\begin{proof}
If $\{a_n\}_{n=1}^\infty$ is Cauchy, then for any $\epsilon >0$, there exists an $N>0$ such that for all $n,\,m \ge N$, $|a_n -a_m| < \epsilon$. If $\epsilon = 1$, there exists an $N_1$ such that for all $m,\,n > N_1$, $|a_n -a_m| < 1$.

Using the triangle inequality, we know that for all $a,\,b,\,c\in\mathbb Z$, the following holds:
\[ \big| |a - c| - |b-c| \big| \le |a - b|. \]
Let $m = N_1 + 1$. By the triangle inequality, for any $n>N_1$, we have:
\begin{align*}
\big| |a_n - a_{N_1} | - |a_n - a_m | \big| \le |a_{N_1} - a_m |,
\end{align*}
But since $m = N_1 + 1$,
\begin{align*}
\big| |a_n - a_{N_1} | - |a_n - a_{(N_1 + 1)} | \big| &\le |a_{N_1} - a_{(N_1+1)} | \\
&<1
\end{align*}


Therefore, $|a_n - a_{N_1}| < |a_n - a_{(N_1+1)}| + 1$, for all $n\ge N_1$.

Let $M=\max\left(|a_{N_1}-a_1|, |a_{N_1}-a_2|, \ldots, |a_{N_1}-a_{N_1}|, |a_{N_1}-a_{(N_1+1)}|\right) + 1$.

Then for all $n\in\mathbb N$, $|a_n - a_{N_1}| \le M$. Hence, $\{a_n\}_{n=1}^\infty$ is bounded by $M$.
\end{proof}

\item Prove that the series
%
$\displaystyle \sum_{n=1}^\infty \frac{1}{n(n+1)}$
%
converges and find its sum.

\medskip

\noindent Hint:  Use induction and the fact that $\frac{1}{n(n+1)}=\frac{1}{n}-\frac{1}{n+1}$ 
to find the partial sums.

\begin{proof}
Let $\{S_N\}_{n=1}^\infty$ be the series of partial sums where $S_N = \sum_{n=1}^N  \frac{1}{n(n+1)}$. We want to show that $S_N \to S$ as $n\to\infty$.

I claim that:
\[ S_N = 1 - \frac 1 {N+1}. \]
We will proceed by induction. When $N=1$, then we have:
\[ \frac 1 {1\cdot(1+1)} = 1- \frac 1 {1+1} = \frac 1 2. \]
Suppose that $S_k = 1 - \frac 1 {k+1}$, we want to show that $S_{k+1} = 1 - \frac 1 {k+2}$. We have that $S_k = 1 - \frac 1 {k+1}$, so $S_{k+1}$ must equal $1 - \frac 1 {k+1} + \frac 1 {(k+1)(k+2)}$. But since we can write $\frac{1}{n(n+1)}$ as $\frac{1}{n}-\frac{1}{n+1}$, we have the following:

\begin{align*}
S_{k+1} &= 1 - \frac 1 {k+1} + \frac 1 {(k+1)(k+2)} \\
&= 1 - \frac 1 {k+1} + \frac{1}{k+1}-\frac{1}{k+2} \\
&= 1- \frac{1}{k+2} \\
&= 1 - \frac 1 {(k+1) + 1}.
\end{align*}
Hence, our assertion that $S_N = 1 - \frac 1 {N+1}$ holds.

Next, we must show that $\{ S_N \}_{N=1}^\infty = \{ 1 - \frac 1 {N+1} \}_{N=1}^\infty$ converges. I claim that $1 - \frac 1 {N+1} \to 1$ as $N \to \infty$.

Fix $\epsilon > 0$. We need to find $M>0$ such that if $N\ge M$,
\[ \left| \left(1 - \frac 1 {N+1}\right) - 1 \right| < \epsilon,\]
or equivalently, since $|1- \frac 1 {N+1} -1 | = | \frac 1 {N+1} | > 0$, $\frac 1 {N+1} < \epsilon$.
By the Archimedean property, there exists $M\in\mathbb N$ such that $M > \frac 1 \epsilon$, but for all $N\in\mathbb N$ such that $N\ge M$, $\frac 1 {N+1} \le \frac 1 M < \epsilon$.
Since this is true for all $\epsilon > 0$, we have that $1 - \frac 1 {N+1} \to 1$ as $N \to \infty$.
\end{proof}


\end{enumerate}

\end{document}
