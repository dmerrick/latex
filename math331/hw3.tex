\documentclass[12pt]{amsart}

\textwidth=1.25\textwidth
\calclayout


\begin{document}

\thispagestyle{empty}

\begin{center}
{\bf Math 331  --- Homework 3 \\
Due:  Wednesday, February 17}
\end{center}

\bigskip

\noindent
Dana Merrick \\
\today

\bigskip

\begin{enumerate}

\setlength{\itemsep}{6pt}

\item Define the set
%
\[ E = \left\{ \frac{n}{n^2+1} : n \in \mathbb{N}\right\}. \]
%
Find $\inf E$.
\begin{proof}
Assuming we're using the $\mathbb N = \{0,1,2,3,\ldots\}$, the first few elements of the set look like this:
\[ E = \left\{ \frac{0}{0^2+1}, \frac{1}{1^2+1}, \frac{2}{2^2+1}, \ldots\right\} = 
\left\{0, \frac 1 2, \frac 2 5,\ldots\right\} \]
I claim $\inf E = 0$. In order for this to be true we need to satisfy two properties: that $0$ is a lower bound of $E$, and that given any other lower bound $x\in E$, $x\le\inf E$. Since $\mathbb N$ is made up of positive numbers, every element of $E$ will be positive. Therefore $0$ is a lower bound of $E$.

To show that $0$ is the greatest lower bound of $E$, we will proceed by contradiction. Assume that there is another lower bound of $E$, call it $y$, such that $y>0$. I claim we have already reached a contradiction, because if $\inf E = y > 0$, then $y$ is not a lower bound because $0\in E$. 

Therefore $0$ must be the greatest lower bound of $E$.
\end{proof}

\item Prove that there exists $x\in \mathbb R$ such that $x^3=2$ and
  show that $x$ is not a rational number.
 
\begin{proof}
First we will show that there exists $x\in \mathbb R$ such that $x^3=2$. Define the set:
\[ E =  \{ y \in \mathbb R : y>0, y^3 < 2 \} \]
Since $1^3 = 1 < 2$, we can say that $1\in E$, so $E\ne\emptyset$. Further, for all $y\in E$, $y^3 < 2 < 8$, so $y < 2$. Therefore $E$ is bounded above, and by the Completeness Axiom, $\sup E$ exists.

Let $x=\sup E$. We claim $x^3 = 2$. We will proceed by contradiction. Suppose $x^3\ne 2$, then either $x^3 < 2$ or $x^3 > 2$.

\bigskip

$x^3 < 2$: Let $\delta = 2-x^3 > 0$. Since $1\in E$, $x$ is an upper bound, so $x\ge 1$. Here $\delta \le 1$. Now consider the value $x+\frac{\delta}{20} > x$. We have the following:
\[
\left(x+\frac{\delta}{20}\right)^3 = 
x^3 + \frac{3}{20} x^2 \delta + \frac{3}{400}x \delta^2 + \frac{1}{8000}\delta^3
\]
Since $\delta \le 1$,
\[ \frac{1}{8000}\delta^3 < \frac 3 {400} \delta^2 < \frac 3 {20} \delta < x^3 + \frac{3}{20} x^2 \delta + \frac{3}{400}x \delta^2 < x^3 + \delta = 2\]
Hence $x+\frac{\delta}{20}\in E$, contradicting the fact that $x$ is an upper bound.

\bigskip

$x^3 > 2$: Let $\epsilon = x^3-2 > 0$. Consider the value $x-\frac{\epsilon}{20}$. Since $x\le 2$, we have the following:
\begin{align*}
\left(x-\frac{\epsilon}{20}\right)^3 &= 
x^3 - \frac{3}{20} x^2 \epsilon + \frac{3}{400}x \epsilon^2 - \frac{1}{8000}\epsilon^3 \\
&> x^3 - \frac{3}{20} x^2 \epsilon + \frac{3}{400}x \epsilon^2 \\
&\ge x^3 - \frac{3}{20} x^2 \epsilon + \frac{3}{400} \epsilon^2 \\
&\ge x^3 - \frac{3}{20} \epsilon \\
&> x^3 - \epsilon = 2
\end{align*}
Now take any $y\in E$. Then $y^3<2<\left(x-\frac{\epsilon}{20}\right)^3$ and so $y<x-\frac{\epsilon}{20}$. Therefore $x-\frac{\epsilon}{20}<x$ is an upper bound of $E$, contradicting the fact that $x$ is the least upper bound.

Therefore in either case we set a contradiction and so we must have that $x^3=2$.

\bigskip

To show that $\sqrt[3] 2$ is irrational, we will proceed by contradiction. Suppose that $\sqrt[3] 2$ is rational, then there exists $a,b \in\mathbb N$ such that $\sqrt[3] 2 = \frac{a}{b}$. Without loss of generality, we can assume that $a$ and $b$ have no common factor. If we cube both sides we get:
\begin{align*}
\left(\sqrt[3] 2\right)^3 &= \left(\frac{a}{b}\right)^3 \\
2 &= \frac{a^3}{b^3}
\end{align*}
After bringing the $b^3$ term to the left side we get:
\[ 2 b^3 = a^3 \]
So we now know that $2|a^3$. Since $2$ is prime, $2|a$. Let $a=2c$ for some $c\in\mathbb N$ Then:
\[ 2b^3 = (2c)^3 = 8c^3 \]
In other words, $b^3 = 4c^3$. But now we have that $2|b^3$, which implies $2|b$, which is a contradiction because $a$ and $b$ have no common factor. Therefore our original supposition is false, and $\sqrt[3] 2$ is irrational.
\end{proof}

\item Show that if $0<r<1$, then for any $\epsilon>0$, there exists $n\in \mathbb{N}$ such that $r^n<\epsilon$.  

Hint:  show that you can write $r^{-1}= 1+s$ for some $s>0$, and then show $(1+s)^n$ can be made arbitrarily large using Bernoulli's inequality.  (Be precise about what arbitrarily large means!)
\begin{proof}
Let $r,\epsilon\in\mathbb R$ such that $0<r<1$ and fix $\epsilon > 0$. We want to show there exists $n\in \mathbb{N}$ such that $r^n<\epsilon$.

First we will take a look at $\frac 1 r$. Notice that since $0<r<1$, $\frac 1 r > 1$. Hence, we can write $\frac 1 r = r^{-1} = 1+s$ for some $s>0$. If we raise both sides of the equation by $n$, we get:
\[ \left(\frac 1 r \right)^n = r^{-n} = (1+s)^n \]
Invoking Bernoulli's inequality, we get:
\begin{align*}
r^{-n} &= (1+s)^n \\
&\ge 1+ns
\end{align*}
In other words:
\[ 
r^n \le \frac 1 {1+ns}
\]
Through the Archimedian property, we know that as $n$ grows to be arbitrarily large, the $\frac 1 {1+ns}$ term will become arbitrarily small. Therefore, there exists some $n$ such that:
\[ \frac 1 {1+ns} < \epsilon \]
Putting all of this together, for some arbitrarily large $n\in\mathbb N$, we get:
\begin{align*}
r^n &\le \frac 1 {1+ns} < \epsilon \\
r^n &< \epsilon
\end{align*}
\end{proof}

\item Given an example to show that the nested interval property need
  not hold if the intervals are open.  Explain where  the proof of the
  nested interval property breaks down if you assume the intervals are open.
\begin{proof}
First, a counterexample. For each $n\in\mathbb N$, let $I_n = (0, \frac 1 n)$. Then $\{I_n\}_{n=1}^\infty$ is a nested sequence of open intervals. Consider some $x\in\mathbb R$. If $x\in I_n$, then $ 0 < x < \frac 1 n$. If we look at some $m\in\mathbb R$ such that $m > \frac 1 x$, then $0 < \frac 1 m < x$. This means that $x\not\in I_m$ if $m > \frac 1 x$. Therefore, there is no $x\in\mathbb R$ that belongs to every $I_n$, so the intersection of every $I_n$ is empty. 

Notice that if we had defined $I_n$ to be closed, then $0$ would be an element of every $I_n$, and their intersection would be $\{0\}$.

The proof of the nested interval property breaks down with open intervals because it works with increasingly smaller intervals. Eventually you will be left with a situation where you have $[x,x]$ for some $x\in\mathbb R$, which works fine because $[x,x]$ is not empty. Conversely, the open interval $(x,x)$ is empty, and this is the reason the property fails for open intervals.
\end{proof}

\item Show that the set 
%
\[ E = \left\{ \frac{2n}{n+1} : n \in \mathbb{N}\right\} \]
%
has an open cover that does not contain a finite subcover.  Show that any open cover of the set $F=E\cup \{\sup E\}$ has a finite subcover.
\begin{proof}
Let $G_n = (0, 1-\frac 1 n)$. I claim $\{G_n\}_{n\in\mathbb N}$ where $n\ge 1$, is an open cover of $E$ which has no finite subcover.

In order for $\{G_n\}_{n\in\mathbb N}$ to be an open cover, it must be true that:
\begin{align*}
E &\subset \bigcup_{n\in\mathbb N} G_n = \left(0,1-\frac 1 1\right)\cup\left(0,1-\frac 1 2\right)\cup\left(0,1-\frac 1 3\right)\cup\ldots \\
E &\subset \left(0,0\right)\cup\left(0,\frac 1 2\right)\cup\left(0,\frac 2 3\right)\cup\ldots
\end{align*}
Fix any $x\in E$. Since every $n\in\mathbb N$ is positive, $x>0$. Since $x>0$, there exists $n\in\mathbb N$ such that $0<x<1-\frac 1 n$. In other words, $x\in(0,1-\frac 1 n) = G_n$.

To show there is no finite subcover, we will proceed by contradiction. Suppose on the contrary that there is a finite subcover. In other words, there exists an $n_1, n_2, \ldots, n_k\in\mathbb N, n_1<n_2<\ldots<n_k$ such that:
\[ E \subset \bigcup_{i=1}^k G_{n_i} \]
If $x\in G_{n_i}$, then $x>0$ and $x<1-\frac 1 {n_i} < 1-\frac 1 {n_k}$. But there exists an $x\in E$ such that $x>1-\frac 1 {n_k}$, which is a contradiction.

\bigskip

Let $F = E\cup \{\sup E\}$, and $A=\{G_a : a \in A\}$ be an open cover of $F$. By the definition of an open cover, some element in $A$ must contain $\sup E$, since $\sup E\in F$.

Since $\sup E\in F$, there exists an $\alpha\in A$ such that $\sup E \in G_\alpha$ where $G_\alpha=(x_\alpha,y_\alpha), x_\alpha < \sup E < y_\alpha$. Therefore there exists an $N\in\mathbb N$ such that if $n\in\mathbb N$ and $n>N$, $\frac 1 n < y_\alpha$.

For each $i\in\mathbb N$, $1\le i < N$, there exists an $\alpha_i\in A$ such that $\frac 1 i \in G_{\alpha_i}$. Now we have:
\[ F \subset (G_\alpha \cup G_{\alpha_1} \cup G_{\alpha_2} \cup \ldots \cup G_{\alpha_{2N-1}}) \]
So $\{G_\alpha : \alpha\in A \}$ contains a finite subcover.
\end{proof}

\end{enumerate} 

\end{document}
