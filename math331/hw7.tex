\documentclass[12pt]{amsart}

\textwidth=1.25\textwidth
\calclayout


\begin{document}

\thispagestyle{empty}

\begin{center}
{\bf Math 331  --- Homework 7 \\
Due:  Monday, April 19}
\end{center}

\bigskip

\noindent
Dana Merrick \\
\today

\bigskip

\begin{enumerate}

\setlength{\itemsep}{6pt}

\item Use the definition of continuity to prove that $f : \mathbb{R} \rightarrow \mathbb{R}$, $f(x)=x^3+1$, is continuous everywhere.

\begin{proof}
Fix $\epsilon >0$ and $x\in\mathbb R$. We want to show that there exists a $\delta > 0$ such that for every $y \in\mathbb R$, if $|x-y| < \delta$, then $| f(x) - f(y) | < \epsilon$.

Let $\delta = \min(1, \tfrac \epsilon {(2|y|+1)^2})$. Fix any $y\in\mathbb R$ such that $|x-y| < \delta$. Since $|x-y| < 1$, $|y| - |x| < 1$, and so $|y| < |x|+1$. Observe that:
%
\[ | f(x) - f(y) | = | x^3 + 1 - y^3 - 1 | = | (x-y) (x^2 + xy + y^2)| \]
%
and
%
\[ | (x-y) (x^2 + xy + y^2)| \le | (x-y) (x^2 + 2xy + y^2)| = | (x-y)(x+y)^2 | \]
%
and finally,
%
\[  | (x-y)(x+y)^2 | < |x-y| \cdot (2|y| +1)^2 < \tfrac \epsilon {(2|y|+1)^2}\cdot (2|y| +1)^2 = \epsilon. \]
\end{proof}

\item Use the definition of continuity to prove that $f : [0,\infty) \rightarrow \mathbb{R}$, $f(x)=\sqrt{x}$, is continuous for all $x\geq 0$.
%
%Hint:  treat the case $x=0$ separately.

\begin{proof}
Fix $\epsilon >0$ and $x \ge 0$. We want to show that there exists a $\delta > 0$ such that for every $y \in [0,\infty)$, if $|x-y| < \delta$, then $| \sqrt x - \sqrt y | < \epsilon$.

We will proceed by cases.

Case 1: $x=0$. Let $\delta = \epsilon^2$. Then if $|x-y| < \delta$,
%
\[ | \sqrt x - \sqrt y | = | \sqrt x - \sqrt y | \cdot \frac {\sqrt x + \sqrt y }{ \sqrt x + \sqrt y } = \frac {|x-y|}{ \sqrt x + \sqrt y},  \]
%
but since $x=0$, we have:
%
\[ \frac {|x-y|}{ \sqrt x + \sqrt y} = \frac y {\sqrt y} = \sqrt y, \]
%
and
%
\[ \sqrt y < \sqrt \delta = \sqrt{\epsilon^2} = \epsilon. \]

Case 2: $x > 0$. Let $\delta = \epsilon\cdot\sqrt x$. Then if $|x-y| < \delta$,
%
\[ | \sqrt x - \sqrt y | = | \sqrt x - \sqrt y | \cdot \frac {\sqrt x + \sqrt y }{ \sqrt x + \sqrt y } = \frac {|x-y|}{ \sqrt x + \sqrt y}, \]
%
and we have that
%
\[ \frac {|x-y|}{ \sqrt x + \sqrt y} \le \frac{|x-y|}{\sqrt x}, \]
%
and
%
\[ \frac{|x-y|}{\sqrt x} < \frac \delta {\sqrt x} = \frac {\epsilon\sqrt x}{\sqrt x} = \epsilon. \]
\end{proof}

\addtocounter{enumi}{1}

%\item Prove, using the definition and the previous problem that if $g : \mathbb{R} \rightarrow \mathbb{R}$ is continuous everywhere and $g(x)\geq 0$ for all $x$, then $G : \mathbb{R} \rightarrow \mathbb{R}$, $G(x)=\sqrt{g(x)}$ is continuous everywhere.
%
%\begin{proof}
%Proof goes here.
%\end{proof}

\item Given a function $f : D \rightarrow \mathbb{R}$, suppose $\displaystyle \lim_{x\rightarrow a} f(x) = L$.  Prove the following lemmas:

\begin{enumerate}
\item There exists $\delta >0$ such that for all for all $x\in D$, $0<|x-a|<\delta$, $|f(x)|<|L|+1$.   

\begin{proof}
Fix $x \in D$. Let $\epsilon=1$. By the definition of a limit, there exists a $\delta$ such that if $0 < | x - a | < \delta$, then $| f(x) - L| < 1$. Since $| f(x) - L | < 1$, we have that $L - 1 < f(x) < L+1$, so therefore $| f(x)| \le \max( |L-1|, |L+1|)$.

Hence, if $| f(x)-L| < 1$, then $| f(x) | < |L| +1$.
\end{proof}

%\item If $L\neq 0$, there exists $\delta>0$ such that for all $x\in D$, $0<|x-a|<\delta$, $|f(x)|>|L|/2$. 
%
%\begin{proof}
%Proof goes here.
%\end{proof}

\end{enumerate}

\item Given functions $f,\, g : D \rightarrow \mathbb{R}$ such that
%
\[ \lim_{x\rightarrow a} f(x) = L  \quad \text{and} \quad \lim_{x\rightarrow a} g(x) = M, \]
%
Prove the following:

\begin{enumerate}

\item $\displaystyle \lim_{x\rightarrow a} (fg)(x) = LM$.

\begin{proof}
Fix $\epsilon > 0$. We want to find a $\delta >0$ such that for all $x\in D$, if $0<|x-a|<\delta$, then $| f(x) \cdot g(x) - LM| < \epsilon$. By problem (4a), there exists a $\delta_0>0$ such that if $0<|x-a|<\delta_0$, then $f(x) \le |L|+1$.

Let $\epsilon_0 = \tfrac \epsilon {2 |L|} > 0$. Then there exists a $\delta_1>0$ where if $0<|x-a|<\delta_1$, then $|f(x) - L| < \tfrac \epsilon {2 |M|}$. Similarly there exists a $\delta_2>0$ where if $0<|x-a|<\delta_2$, then $|g(x) - M| < \tfrac \epsilon {2 (|L|+1)}$. Let $\delta = min( \delta_0, \delta_1, \delta_2)$, then if $0< |x-a| < \delta$, then $|f(x)| \le |L|+1$ and $| f(x) - L| < \tfrac \epsilon {2 |L|}$ and $| g(x) - M| < \tfrac \epsilon {2(|L|+1)}$.

Observe that
%
\[ | f(x) \cdot g(x) - LM | = | f(x) \cdot g(x) - f(x)M + f(x)M - LM |. \]
%
By the triangle inequality,
%
\[  | f(x) \cdot g(x) - LM | \le  f(x) \cdot g(x) - f(x)M| + |f(x)M - LM| \]
%
or
%
\[  | f(x) \cdot g(x) - LM | \le  |f(x)| \cdot |g(x) - M| + |M| \cdot |f(x) - L|. \]
But now we can say:
%
\[ |f(x)| \cdot |g(x) - M| + |M| \cdot |f(x) - L| < (|L|+1) \frac \epsilon {2(|L|+1)} + |M| \frac \epsilon {2|M|}, \]
%
and
%
\[ (|L|+1) \frac \epsilon {2(|L|+1)} + |M| \frac \epsilon {2|M|} = \epsilon. \]
\end{proof}

%\item If $L\neq 0$, there exists $\gamma>0$ such that $\frac{f}{g}$ is defined on $B_\gamma^0(a)$ and 
%$\displaystyle \lim_{x\rightarrow a}\frac{f}{g}(x) = \frac{L}{M}$. 
%
%\begin{proof}
%Proof goes here.
%\end{proof}

  \end{enumerate}

%\item Use the definition of the limit to prove that
%%
%\[ \lim_{x\rightarrow 4} \frac{\sqrt{x}-2}{x-4} = \frac{1}{4}. \]
%
%\begin{proof}
%Proof goes here.
%\end{proof}


%\item (Extra Credit) Prove that for any $n\in \mathbb{N}$, $ f : [0,\infty) \rightarrow \mathbb{R}$, $f(x)=x^{1/n}$ is continuous everywhere.
%
%Hint:  Use the identity 
%%
%\[ a^n - b^n  = (a-b)\sum_{k=0}^{n-1} a^kb^{n-1-k}. \]
%
%\begin{proof}
%Proof goes here.
%\end{proof}
%
%\item (Extra Credit) Prove that the function $Q : \mathbb{R} \rightarrow \mathbb{R}$,
%%
%\[ Q(x) =  \begin{cases}
%0 & x \in \mathbb{R}\setminus \mathbb{Q} \\
%1/q & x= p/q, \; p,\,q \in \mathbb{Z}, \, p\neq 0, \, q>0, \,(p,q)=1 \\
%1 & x = 0,
%\end{cases}
%\]
%%
%is continuous at each irrational $x$ and discontinuous at each rational $x$.
%
%Hint:  The proof for $x$ rational is sketched in the notes.  For $x$ irrational, fix $\epsilon>0$, and choose $N$ such that $N^{-1} < \epsilon$.  Show that if we let
%%
%\[ \delta = \inf\{ |x-p/k| : p \in \mathbb{Z}, k\in \mathbb{N}, k\leq N \}, \]
%%
%then $\delta >0$, and for any $y\in \mathbb{Q}$ such that $|x-y|<\delta$, $|Q(x)-Q(y)|<\epsilon$. 
%
%\begin{proof}
%Proof goes here.
%\end{proof}


\end{enumerate}



\end{document}
