\documentclass[12pt]{amsart}

\textwidth=1.25\textwidth
\calclayout


\begin{document}

\thispagestyle{empty}

\begin{center}
{\bf Math 400  --- Examination No.\ 2 \\
Due:  Friday, May 11th, by 4:00PM}
\end{center}

\bigskip

\noindent
Dana Merrick \\
\today

\bigskip

\begin{enumerate}\setlength{\itemsep}{6pt}

\item {\it Fibonacci Numbers}

Recall that the sequence of Fibonacci numbers $\{ F_i \}_{i=0}^\infty$ is defined recursively as follows:
%
\[ F_1 = F_2 = 1\]
%
\[ F_n = F_{n-1} + F_{n-2},\, n\ge 3. \]
%
\begin{enumerate}\setlength{\itemsep}{6pt}
\item
\begin{enumerate}\setlength{\itemsep}{6pt}
\item Prove that for $n\ge 1,$ $F_n$ and $F_{n+1}$ are {\it relatively prime}, that is
%
\[\gcd(F_n, F_{n+1}) = 1. \]
%
Recall that for $a,\,b \in \mathbb Z$, not both zero, $\gcd(a,b)=1$ if and only if there exist $s,\,t \in \mathbb Z$ such that $1=sa+tb$.

\begin{proof}
We will use the Euclidean algorithm to find $\gcd(F_n, F_{n+1})$. We have:

\begin{align*}
F_{n+1} &= 1 \cdot F_n + F_{n-1} \\
F_n &= 1 \cdot F_{n-1} + F_{n-2} \\
&\vdots \\
F_4 &= 1 \cdot F_3 + F_2 \\
F_3 &= 2 \cdot F_2 + 0
\end{align*}

Substituting in $F_4=3$, $F_3=2$, $F_2=1$, and $F_1=1$, we get:

\begin{align*}
3 &= 1\cdot2 + 1 \\
2 &= 2\cdot1 + 0.
\end{align*}

Hence, $\gcd(F_n, F_{n+1}) =1$.
\end{proof}

\item Using 1(a)(i), prove that for $n\ge 1$, $F_n$ and $F_{n+2}$ are relatively prime.

\begin{proof}
Consider two Fibonacci numbers $F_n$ and $F_{n+2}$. By the definition of the Fibonacci sequence, we have that,
%
\[ F_{n+2} = F_n \cdot F_{1} + F_{2} \cdot F_{n-1}. \]

From this, we can see that,
%
\begin{align*}
\gcd(F_n, F_{n+2}) &= \gcd(F_n, F_n \cdot F_{1} + F_2 \cdot F_{n-1}) \\
&= \gcd(F_n, F_n\cdot 1 + 1\cdot F_{n-1}) \\
&= \gcd(F_n, F_{n-1})
\end{align*}
Since by 1(a)(i), $F_n$ and $F_{n-1}$ are relatively prime.

Hence, by 1(a)(i), $\gcd(F_n, F_{n-1}) = \gcd(F_n, F_{n+2}) = 1$.
\end{proof}

\end{enumerate}

\item In class we observed the following about 4 consecutive Fibonacci numbers:
\begin{itemize}\setlength{\itemsep}{6pt}
\item Let $x$ be the product of the first and fourth numbers.
\item Let $y$ be {\it twice} the product of the second and third numbers.
\item Let $z$ be the {\it sum} of the square of the second number with the square of the third number.
\end{itemize}
Then $x^2 + y^2 = z^2$, that is, $(x,y,z)$ is a Pythagorean triple.

For example, for $F_1, F_2, F_3$, and $F_4$, we have:
\begin{itemize}\setlength{\itemsep}{6pt}
\item $x=F_1 \cdot F_4 = 1 \cdot 3 = 3$.
\item $y=2 \cdot F_2 \cdot F_3 = 2\cdot 1 \cdot 2 = 4$.
\item $z=F_2^2 \cdot F_3^2 = 1^2 \cdot 2^2 = 5$.
\end{itemize}

\bigskip

Please complete the following:

\begin{enumerate}\setlength{\itemsep}{6pt}
\item Express the procedure described above in the form of a claim that can be proven via {\it mathematical induction} and then prove the claim by induction.

\subsection*{Claim}
If $F_n$ is a Fibonacci number, then,
%
\[ (F_n\cdot F_{n+3}, 2F_{n+1} \cdot F_{n+2}, F_{n+1}^2 + F_{n+2}^2)\]
%
is a Pythagorean triple.

\begin{proof}
We will proceed by induction on $n$.

Base case: When $n=1$, we know that $F_1 = F_2 = 1$, $F_3 = 2$, and $F_4 = 3$. So we have:
%
\begin{align*}
(F_1\cdot F_{4}, 2F_{2}\cdot F_{3}, F_{2}^2 + F_{3}^2) &= (1\cdot 3, 2\cdot 1 \cdot 2, 1^2 + 2^2) \\
&= (3, 4, 5)
\end{align*}
%
Which is clearly a Pythagorean triple, since $3^2+4^2 = 5^2$.

\bigskip

Inductive case: Assume the claim holds for $F_k$. We want to show it also holds for $F_{k+1}$. In other words, assuming that,
%
\[ (F_k\cdot F_{k+3}, 2F_{k+1} \cdot F_{k+2}, F_{k+1}^2 + F_{k+2}^2)\]
%
is a Pythagorean triple, we want to show that,
%
\[ (F_{k+1}\cdot F_{k+4}, 2F_{k+2} \cdot F_{k+3}, F_{k+2}^2 + F_{k+3}^2)\]
%
is also a Pythagorean triple.

This means that we need to show:
%
\[ (F_{k+1}\cdot F_{k+4})^2 + (2F_{k+2} \cdot F_{k+3})^2 = (F_{k+2}^2 + F_{k+3}^2)^2. \]
%

Observe that:
\begin{align*}
(F_{k+1}\cdot F_{k+4})^2 + (2F_{k+2} \cdot F_{k+3})^2 &= \ldots \\
\\
&= F_{k+2}^4 + 2\cdot F_{k+2}^2 \cdot F_{k+3}^2 + F_{k+3}^4 \\
&= (F_{k+2}^2 + F_{k+3}^2)^2
\end{align*}
\end{proof}

\item Under what conditions will $(x,y,z)$ form a {\it primitive} Pythagorean triple? Your answer should be as specific as possible and should be connected to your claim in 1(b)(i).

\subsection*{Solution}
We know that Pythagorean triples are generated by $(t^2 - s^2, 2st, t^2 + s^2)$, for all $s<t$. Similarly, we proved that a Pythagorean triple if one of $s,\,t$ is even and the other is odd.

Since in the definition of Fibonacci Pythagorean triples the second term is $2\cdot F_{n+1}\cdot F_{n+2}$, we can say that $s = F_{n+1}$ and $t=F_{n+2}$. Therefore, a Pythagorean triple generated this way is primitive if $F_{n+1}$ and $F_{n+2}$ are not both odd.

In other words, a Pythagorean triple generated by $F_n$ is primitive if:
\[ n\mod 3 \not= 0. \]

\end{enumerate}

\end{enumerate}

\item {\it Defn.} Let $(a,b,c)$ be a Pythagorean triple. We say that triple $(a,b,c)$ is a FP triple if and only if $(a,b,c)$ is produced by the procedure described 1(b)(i). Moreover for $n\ge 1$, let $FP(n)$, denote the FP-triple created by $F_n$, $F_{n+1}$, $F_{n+2}$, and $F_{n+3}$.

Thus, $FP(1) = (3,4,5)$ and $FP(2) = (5,12,13)$.

{\it Defn.} The FP triple $(a,b,c)$ is PFP if and only if $(a,b,c)$ is a {\it primitive} FP triple.

{\it Defn.} For $n \ge 1$, let,
%
\[ C_n = \{ FP(i) : 1 \le i \le n \} \]
%
and,
%
\[ D_n = \{ FP(i) : 1 \le i \le n \textrm{  and } FP(i) \textrm{ is } PFP \} \]
%
Answer the following:
\begin{enumerate}\setlength{\itemsep}{6pt}
\item What is $|C_n|$, the cardinality of $C_n$?

\subsection*{Solution}
By its definition, $|C_n| = n$.

\item Give a formula for $|D_n|$.

\subsection*{Solution}
\[ |D_n| = \left\lfloor \frac{2(n+1)}{3} \right\rfloor \]
\item Determine:
%
\[ \lim_{n\to\infty} \frac{|D_n|}{|C_n|} \]
%

\subsection*{Solution}
%
\[ \lim_{n\to\infty} \frac{|D_n|}{|C_n|} =  \lim_{n\to\infty} \frac {\left\lfloor \frac{2(n+1)}{3} \right\rfloor}{n} = \frac 2 3 \]
%

\end{enumerate}

\item In class we considered the sequence of Fibonacci numbers {\it modulo} 10 and saw that this sequence repeated a pattern after 60 terms.

\begin{enumerate}\setlength{\itemsep}{6pt}
\item If we consider the sequence of Fibonacci numbers {\it modulo} 8, what is the length of its repeated pattern?

\subsection*{Solution}
The pattern ($\{1, 1, 2, 3, 5, 0, 5, 5, 2, 7, 1, 0\}$) has length 12.

\item 
\begin{enumerate}\setlength{\itemsep}{6pt}
\item What is the smallest $n$ such that 30 divides $F_n$?

\subsection*{Solution}
1548008755920, the 60th Fibonacci number, is the smallest such that 30 divides it.

\item If we consider the sequence of Fibonacci numbers {\it modulo} 30, what is the length of its repeated pattern?

\subsection*{Solution}
The length of the repeated pattern is 60.

\end{enumerate}

\item We know that $F_{15} = 610$. If we consider the sequence of Fibonacci numbers {\it modulo} 610, what is the length of its repeated pattern?

\subsection*{Solution}
The pattern also has length 60.

\end{enumerate}

\item
\begin{enumerate}\setlength{\itemsep}{6pt}

\item A shopper spends a total of \$5.49 for oranges, which cost \$0.18 a piece, and grapefruits, which cost \$0.33 each. What is the {\it minimum} number of pieces of fruit that the shopper could have bought?

\subsection*{Solution}
Consider the non-negative integer solutions to the following equation:
%
\[ 5.49 = 0.18 O + 0.33 G \]
%
The solutions are as follows:
\bigskip
\begin{itemize}
\item 3 oranges, 15 grapefruit.
\item 14 oranges, 9 grapefruit.
\item 25 oranges, 3 grapefruit.
\end{itemize}

\bigskip

The first solution has 18 total fruit, which is the minimum number of pieces that the shopper could have bought.

\item An ancient chinese puzzle found in the 6th century work of the mathematician Chang Ch'iu-chien, called the ``Hundred Foods'' Problem asks:

If a cock is worth 5 coins, a hen 3 coins, and 3 chickens together are worth 1 coin, how many cocks, hens, and chickens totaling 100 can be bought for 100 coins?

\subsection*{Solution}
Consider the non-negative integer solutions to the following equation:
%
\[ 100 = 5k + 3h + c = k + h + 3c \]
%

\bigskip

The solutions are as follows:
\bigskip
\begin{itemize}
\item 0 cocks, 25 hens, 75 chickens.
\item 4 cocks, 18 hens, 78 chickens.
\item 8 cocks, 11 hens, 81 chickens.
\item 12 cocks, 4 hens, 84 chickens.
\end{itemize}

\end{enumerate}

\item Let $(R, +, \cdot)$ be a commutative ring.

\bigskip

\begin{enumerate}\setlength{\itemsep}{6pt}
\item Prove that if $I_1$ and $I_2$ are {\it ideals} of $R$, then $I_1 \cap I_2$ is an ideal of $R$.

\begin{proof}
First we must show that $I_1 \cap I_2$ is an ideal of $R$. We know from a previous theorem that the intersection of two subgroups is a subgroup, so the intersection of $I_1$ and $I_2$ forms a subgroup.

Let $x \in I_1 \cap I_2$. If $y \in R$, then $x+y$ and $y+x$ are in both $I_1$ and $I_2$ and thus in $I_1 \cap I_2$. So $I_1 \cap I_2$ is an ideal of $R$.
\end{proof}

\item Let $A \subseteq R$.

\bigskip

{\it Defn.} The Annihilator of $A$, defined by,
%
\[ Ann(A) = \{ x\in R : a\cdot x = 0_R\,\, \forall a \in A \} \]
%
is a subset of $R$.

\bigskip

\begin{enumerate}\setlength{\itemsep}{6pt}
\item Prove that $Ann(A)$ is an {\it ideal} of $R$.

\begin{proof}
If $m,\,n\in Ann(A)$, then so are $m-n$ and $rm$ for all $r\in R$. Therefore $Ann(A)$ is a subring of $R$. By the distributive law, $Ann(A)$ is closed under addition and right multiplication.

Fix $x\in Ann(A)$ and $r \in R$. Select any $a \in A$. Then $ar \in A$, but then $(ar)x = 0$ because $x\in Ann(A)$. Therefore, $a(rx) = 0$ and $rx \in Ann(A)$. Thus, $Ann(A)$ is an ideal of $R$.
\end{proof}

\item Suppose that,
%
\[ R = \mathbb Z_4 \times \mathbb Z_{100},\]
%
and,
%
\[ Y = \{ (0,52), (2,40) \}. \]

Determine $Ann(Y)$.

\subsection*{Solution}
%
\[ Ann(Y) = \{ (0,0), (0,25), (0,50), (0,75), (2,0), (2,25), (2,50), (2,75) \} \]
%

%Here is the set of $r\in R$ such that $r \cdot (0,52) = (0,0)$:
%%
%\[ \{ (0,0), (0,25), (0,50), (0,75), (2,0), (2,25), (2,50), (2,75) \} \]
%%
%Here is the set of $r\in R$ such that $r \cdot (2,40) = (0,0)$:
%%
%\[ \{ (0,0), (0,5), (0,10), (0,20), (0,25), (0,30), (0,20), (0,40), (0,50), (0,55), (0,70), (0,75), (0,90) \} \]
%\[ \{ (2,0), (2,5), (2,10), (2,20), (2,25), (2,30), (2,20), (2,40), (2,50), (2,55), (2,70), (2,75), (2,90) \} \]
%%

\item Suppose that $R = \mathbb C$ (where $\mathbb C$ is the ring of complex numbers with the usual addition and multiplication), and let,
%
\[ W = \{ 1 + i, \sqrt 2 - i \} \]
%
Determine $Ann(W)$.

\subsection*{Solution}
%
\[ Ann(W) = \{ 0 \} \]
%

\end{enumerate}
\end{enumerate}

\item Let $(R, +, \cdot)$ be a ring.
\begin{enumerate}\setlength{\itemsep}{6pt}
\item Define $P : R[x] \to R$ as follows:
%
\[ P(a_0 + a_1 x + a_2 x^2 + \ldots + a_n x^n) = a_0 \]
%
\begin{enumerate}\setlength{\itemsep}{6pt}
\item Prove that P is a {\it ring homomorphism} from $R[x]$ {\it onto} $R$.

\begin{proof}
We must show that:
%
\[ P(a + b) = P(a) + P(b) \]
%
and,
%
\[ P(ab) = P(a)P(b) \]
%
For all $a,\,b \in R$.

\bigskip

Let
%
\[ a_0 + a_1 x + \ldots + a_n x^n \in R[x], \]
%
and,
%
\[ b_0 + b_1 x + \ldots + b_n x^n \in R[x]. \]
%
Then we have:
%
\[ P(a_0 + a_1 x + \ldots + a_n x^n + b_0 + b_1 x + \ldots + b_n x^n) \]
\[ = P[ (a_0+b_0) + (a_1+b_1)x + \ldots + (a_n + b_n)x^n] = a_0 + b_0 \]
\[ = P(a_0 + a_1 x + \ldots + a_n x^n) + P(b_0 + b_1 x + \ldots + b_n x^n). \]
%
Also, we have:
%
\[ P[ (a_0 + a_1 x + \ldots + a_n x^n) \cdot (b_0 + b_1 x + \ldots + b_n x^n) ] \]
\[ = P[ (a_0\cdot b_0) + (a_1\cdot b_0 + a_0\cdot b_1)x + \ldots + (a_n\cdot b_0 + a_{n-1}\cdot b_1 + \ldots + (a_0\cdot b_n)x^n ] \]
\[ = a_0 \cdot b_0  = P(a_0 + a_1 x + \ldots + a_n x^n) \cdot P(b_0 + b_1 x + \ldots + b_n x^n). \]
%
Finally, it is clear that the multiplicative identity in $R[x]$ maps to the multiplicative identity in $R$, since $P(1 + x + x^2 + \ldots + x^n)=1$.
\end{proof}

\item What is $Ker(P)$?

\subsection*{Solution}
$Ker(P) = \{ a_0 + a_1 x + \ldots + a_n x^n \in R[x] : a_0 = 0\}$

\item To what ring is $R[x] / Ker(P)$ isomorphic?

\subsection*{Solution}
By the first isomorphism theorem, the image of $R[x]$ under $P$ is isomorphic to $R[x] / Ker(P)$.

\end{enumerate}

\item Consider the derivative mapping:
%
\[ D : R[a] \to R[a] \]
%
given by,
%
\[ D(a_0 + a_1 x + \ldots + a_n x^n) = a_1 + 2a_2 x + \ldots + n a_n x^{n-1} \]
%
\begin{enumerate}\setlength{\itemsep}{6pt}
\item Prove $D$ is a {\it group} homomorphism.

\begin{proof}
We must show that:
%
\[ D(a + b) = D(a) + D(b) \]
%
and,
%
\[ D(ab) = D(a)D(b) \]
%
For all $a,\,b \in R[a]$.

\bigskip

Let
%
\[ a_0 + a_1 x + \ldots + a_n x^n \in R[a], \]
%
and,
%
\[ b_0 + b_1 x + \ldots + b_n x^n \in R[a]. \]
%
Then we have:
%
\[ D(a_0 + a_1 x + \ldots + a_n x^n + b_0 + b_1 x + \ldots + b_n x^n) \]
\[ = D[ (a_0+b_0) + (a_1+b_1)x + \ldots + (a_n + b_n)x^n] \]
\[ = (a_1+b_1) + 2(a_2+b_2)x + \ldots + n(a_n+b_n)x^n \]
\[ = (a_1 + 2a_2x + \ldots + na_nx^n) + (b_1 + 2b_2x + \ldots + nb_nx^n)  \]
\[ = D(a_0 + a_1 x + \ldots + a_n x^n) + D(b_0 + b_1 x + \ldots + b_n x^n). \]
%
Also, we have:
%
\[ D[ (a_0 + a_1 x + \ldots + a_n x^n) \cdot (b_0 + b_1 x + \ldots + b_n x^n) ] \]
\[ = D[ (a_0\cdot b_0) + (a_1\cdot b_0 + a_0\cdot b_1)x + \ldots + (a_n\cdot b_0 + a_{n-1}\cdot b_1 + \ldots + (a_0\cdot b_n)x^n ] \]
\[ = (a_1\cdot b_0 + a_0\cdot b_1) + 2(a_2\cdot b_0 + a_1\cdot b_1 + a_0\cdot b_2)x + \ldots + n(a_n\cdot b_0 + a_{n-1}\cdot b_1 + \ldots + a_0\cdot b_n)x^n \]
\[ = (a_1 + 2a_2 x + \ldots + n a_n x^{n-1} ) \cdot (b_1 + 2b_2 x + \ldots + n b_n x^{n-1}) \]
\[ = D(a_0 + a_1 x + \ldots + a_n x^n) \cdot D(b_0 + b_1 x + \ldots + b_n x^n). \]
\end{proof}

\item What is $Ker(D)$?

\subsection*{Solution}
$Ker(D) = \{ a_0 + a_1 x + \ldots + a_n x^n \in R[a] : a_i = 0 \,\forall i \ge 1\}$

\item Explain why $D$ is {\it not} a {\it ring} homomorphism.

\subsection*{Solution}
$D$ is not a ring homomorphism because there is no mapping from the multiplicative identity in $R[a]$ to itself.

\item Is $D : \mathbb Z[a] \to \mathbb Z[a]$ {\it onto}? Justify your answer.

\subsection*{Solution}
No. Recall that for $D$ to be onto it must be true that for all $b\in R[a]$ there is an $a\in R[a]$ such that $D(a)=b$. Consider $b\in R[a]$ such that the $x^1$ term is odd. This element can never be mapped to by $D$, since by the definition of $D$ the $x^1$ term must be even (since it is multiplied by 2).

\end{enumerate}

\end{enumerate}

\item Refer to Section 3.8 on page 149 of your handout.

Find the greatest common divisor in $\mathbb Z[i]$ of $11 + 7i$ and $18-i$. {\it Show work!}

{\it Hint:} Read carefully the proof of Theorem 3.8.1.

\subsection*{Solution}
Recall the $d$ function for the Gaussian integers is defined by:
\[ d( a + bi ) = a^2 + b^2 = (a+bi)(a-bi) \]

For $11+7i$ we have:
\[ d(11+7i) = 11^2 + 7^2 = 170 = 2\cdot 5\cdot 17\]
\[ 11 + 7i = -i \cdot (1+i) \cdot (1+2i) \cdot (4+i) \]

For $18-i$ we have:
\[ d(18 - i) = 10^2 + 1^2 = 325 = 5^2 \cdot 13 \]
\[ 18-i = -i \cdot (2+i)^2 \cdot (3+2i) \]

So the greatest common divisor in $\mathbb Z[i]$ is $-i$. But recall that the units of $\mathbb Z[i]$ are $\{ 1, -1, i, -i \}$. Therefore, $11+7i$ and $18-i$ are relatively prime.

\end{enumerate}% top level

\end{document}
