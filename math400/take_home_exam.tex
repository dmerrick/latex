\documentclass[12pt]{amsart}

\textwidth=1.25\textwidth
\calclayout


\begin{document}

\thispagestyle{empty}

\begin{center}
{\bf Math 400  --- Take Home Exam \\
Due:  Friday, March 19}
\end{center}

\bigskip

\noindent
Dana Merrick \\
\today

\bigskip

\begin{enumerate}\setlength{\itemsep}{6pt}

\item
\begin{enumerate}\setlength{\itemsep}{6pt}

\item Consider the quantity $x$ represented in base ten fractional from by $\frac 7 9$.
\begin{enumerate}\setlength{\itemsep}{6pt}
\item Represent $x$ as a {\it finite} sum of {\it distinct} unit fractions.
\subsection*{Solution}
$\frac 1 2 + \frac 1 4 + \frac 1 {36}$

\item Represent $x$ in base 3 form.
\subsection*{Solution}
$(0.21)_3$

\item Represent $x$ in base 8 form.
\subsection*{Solution}
$(0.\overline{61})_8$

\item Represent $x$ in base 16 form.
\subsection*{Solution}
$(0.\overline{\text{C}71})_{16}$

\end{enumerate}

\item Prove that every positive integer can be {\it uniquely} represented as the sum or difference of powers of 3 (e.g. $11=3^3-3^2-3^1-3^0$).

Hint: Consider the base 3 representation of integers and induct on $p$, the {\it number} of symbols in the base 3 representation.
\begin{proof}
Assume that every positive integer $n$ can be uniquely represented as the sum or difference of powers of 3. We will proceed by induction on $p$, the number of symbols in the base 3 representation of $n$.

Notice that any time we're given 2 as a coefficient in the sum/difference of powers of 3 representation, we can rewrite it as $2 = 3 -1$. Therefore, for any $i\in\mathbb Z$ we have that:
\[ 2 \cdot 3^i = 3\cdot 3^i - 1\cdot 3^i = 3^{i+1} - 3^i. \]
This allows us to ``shift left" the base 3 representation of $n$.

As a base case, the proposition is true because when $p=1$, we find that $1=3^0$. Next, assume that the proposition is true for $p=k$. We want to show that the supposition is true for $p=k+1$.

In other words, $p=k+1$ implies we are adding another position to the base 3 representation. We really only care if the new position contains a 2. If the {k+1}th position is occupied by a 2, then $2\cdot 3^{k+1}$ is in the base 3 representation. However, since $2\cdot 3^{k+1} = 3^{k+2} - 3^{k+2}$, we can convert any 2 we find such that every digit in the base 3 representation is either 0 or $\pm 1$.

Hence, every $n$ can be uniquely represented as the sum or difference of powers of 3.
\end{proof}

\end{enumerate}

\item The algebraic structure $(\mathbb Z_3, +, \cdot)$ is the set $\{0,1,2\}$ together with the operations of arithmetic (addition and multiplication) modulo 3. This structure is an example of a (finite) field. The vector space $V = (\mathbb Z_3)^3$ is a vector space over the field $\mathbb Z_3$. $V$ has $3^3=27$ vectors, 26 of which are nonzero. The dimension of $V$ is 3, and all of the results associated with vector spaces apply.

There are 26 letters in our alphabet, and we may correspond each letter to a nonzero vector in $V$ according to the letter's position. Thus,
\[A \leftrightarrow \begin{pmatrix}
0\\
0\\
1\end{pmatrix}\]
\[ \textrm{and} \]
\[Z \leftrightarrow \begin{pmatrix}
2\\
2\\
2\end{pmatrix} \]

The usual basis $B = \{\epsilon_1, \epsilon_2, \epsilon_3 \}$ corresponds to letters I, C, and A respectively.

\begin{enumerate}\setlength{\itemsep}{6pt}

\item Consider the mapping $\varphi : (\mathbb Z_3 )^3 \to (\mathbb Z_3 )^3$ given by:
\[\varphi\left(\langle a_1, a_2, a_3 \rangle\right) = \langle a_1+a_2, a_2+a_3, a_1+a_3 \rangle \]

\begin{enumerate}\setlength{\itemsep}{6pt}

\item Show $\varphi$ is {\it linear}.
\begin{proof}
Recall that $\varphi$ is linear if for any two vectors $\alpha,\,\beta \in (\mathbb Z_3 )^3$ and a constant $c$, the following two conditions are satisfied:
\[ \varphi( \alpha + \beta) = \varphi(\alpha) + \varphi(\beta) \]
\[ \textrm{and} \]
\[ \varphi(c\cdot\alpha) = c\cdot\varphi(\alpha). \]
Let $\alpha = \langle a_1, a_2, a_3 \rangle$ and $\beta = \langle b_1, b_2, b_3 \rangle$. Then we have the following:
\begin{align*}
\varphi(\alpha) + \varphi(\beta) &= \langle a_1 + a_2, a_2 + a_3, a_1 + a_3 \rangle +
\langle b_1 + b_2, b_2 + b_3, b_1 + b_3 \rangle \\
&= \langle a_1 + b_1 + a_2 + b_2, a_2 + b_2 + a_3 + b_3, a_1 + b_1 + a_3 + b_3 \rangle \\
&= \varphi(\langle a_1 + b_1 \rangle, \langle a_2 + b_2 \rangle, \langle a_3 + b_3 \rangle) \\
&= \varphi(\alpha + \beta)
\end{align*}
\[ \textrm{and} \]
\begin{align*}
c \cdot \varphi(\alpha) &= c \cdot \langle a_1 + a_2, a_2 + a_3, a_1 + a_3 \rangle \\
&= \langle c\cdot(a_1 + a_2), c\cdot(a_2 + a_3), c\cdot(a_1 + a_3) \rangle \\
&= \varphi(c\cdot\alpha).
\end{align*}
Therefore, $\varphi$ is linear.
\end{proof}

\item Express $\varphi$ in {\it matrix} form $A=(a_{ij})$.
\subsection*{Solution}
\[ A \cdot \begin{pmatrix} a_1 \\ a_2 \\ a_3 \end{pmatrix}
= \begin{pmatrix} a_1 + a_2 \\ a_2 + a_3 \\ a_1 + a_3 \end{pmatrix} \]
\[ A = \begin{pmatrix} 1&1&0 \\ 0&1&1 \\ 1&0&1 \end{pmatrix} \]

\item Compute $\det (A)$.
\subsection*{Solution}
\[ \det(A) = 1 + 1 + 0 - 0 - 0 - 0 = 1+1 = 2 \]

\item Explain why $\varphi$ is {\it nonsingular}.
\subsection*{Solution}
Recall that a matrix is nonsingular if and only if its determinate is nonzero. Since $\det (A) = 2$, $\varphi$ is nonsingular.

\item Since A is nonsingular, it is {\it invertible}. Determine $A^{-1}$.
\subsection*{Solution}
\[ \begin{pmatrix} 1&1&0 \\ 0&1&1 \\ 1&0&1 \end{pmatrix} A^{-1}
= \begin{pmatrix} 1&0&0 \\ 0&1&0 \\ 0&0&1 \end{pmatrix} \]
\[  A^{-1} = \begin{pmatrix}
\dfrac 1 2 & -\dfrac 1 2 & \dfrac 1 2 \\
\dfrac 1 2 & \dfrac 1 2 & -\dfrac 1 2 \\
-\dfrac 1 2 & \dfrac 1 2 & \dfrac 1 2
 \end{pmatrix} \]
 
\end{enumerate}

\item With respect to the usual basis and the {\it natural} correspondence $\alpha$ between letters and $V$, determine:
\[ \lambda(\textrm{MATHEMATICS}) \textrm{, where} \]
\[ \lambda = \alpha^{-1} \circ \varphi \circ \alpha \]
\begin{enumerate}\setlength{\itemsep}{6pt}

\item Give the {\it permutation} of letters associated with $\lambda$.
\subsection*{Solution}
$\lambda(\textrm{MATHEMATICS}) = \textrm{ZDYWKZDYJLU}$

\end{enumerate}
\item Consider the {\it permutation} $\Pi$ of letters that {\it reverses} letters, zyx$\ldots$a.
\begin{enumerate}\setlength{\itemsep}{6pt}

\item What is the {\it order} of $\Pi$?
\subsection*{Solution}
The order of $\Pi$ is 2, since $\Pi(\Pi(\gamma)) = \gamma$ for every $\gamma$.

\item Is the mapping
\[ \alpha \circ \Pi \cdot \circ^{-1} : (\mathbb Z_3 )^3 \to (\mathbb Z_3 )^3 \]
one-to-one? Justify your answer.
\subsection*{Solution}
Yes. We know that $\alpha$ is one-to-one, which in turn implies that $\alpha^{-1}$ is one-to-one. Since $\Pi$ is a permutation, $\Pi$ is also one-to-one. Finally, the composition of one-to-one functions is always one-to-one.

\item Give the image of each nonzero vector in $(\mathbb Z_3 )^3$ under $\alpha \circ \Pi \circ \alpha^{-1}$.
\subsection*{Solution}

\begin{align*}
\alpha \circ \Pi \circ \alpha^{-1}(A) &= Z \\
\alpha \circ \Pi \circ \alpha^{-1}(B) &= Y \\
\alpha \circ \Pi \circ \alpha^{-1}(C) &= X \\
\alpha \circ \Pi \circ \alpha^{-1}(D) &= W \\
\alpha \circ \Pi \circ \alpha^{-1}(E) &= V \\
\alpha \circ \Pi \circ \alpha^{-1}(F) &= U \\
\alpha \circ \Pi \circ \alpha^{-1}(G) &= T \\
\alpha \circ \Pi \circ \alpha^{-1}(H) &= S \\
\alpha \circ \Pi \circ \alpha^{-1}(I) &= R \\
\alpha \circ \Pi \circ \alpha^{-1}(J) &= Q \\
\alpha \circ \Pi \circ \alpha^{-1}(K) &= P \\
\alpha \circ \Pi \circ \alpha^{-1}(L) &= O \\
\alpha \circ \Pi \circ \alpha^{-1}(M) &= N
\end{align*}
% splitting it up to encourage a page break
\begin{align*}
\alpha \circ \Pi \circ \alpha^{-1}(N) &= M \\
\alpha \circ \Pi \circ \alpha^{-1}(O) &= L \\
\alpha \circ \Pi \circ \alpha^{-1}(P) &= K \\
\alpha \circ \Pi \circ \alpha^{-1}(Q) &= J \\
\alpha \circ \Pi \circ \alpha^{-1}(R) &= I \\
\alpha \circ \Pi \circ \alpha^{-1}(S) &= H \\
\alpha \circ \Pi \circ \alpha^{-1}(T) &= G \\
\alpha \circ \Pi \circ \alpha^{-1}(U) &= F \\
\alpha \circ \Pi \circ \alpha^{-1}(V) &= E \\
\alpha \circ \Pi \circ \alpha^{-1}(W) &= D \\
\alpha \circ \Pi \circ \alpha^{-1}(X) &= C \\
\alpha \circ \Pi \circ \alpha^{-1}(Y) &= B \\
\alpha \circ \Pi \circ \alpha^{-1}(Z) &= A
\end{align*}
\end{enumerate}

\item Is $\alpha \circ \Pi \circ \alpha^{-1}$ {\it linear}? Justify your answer.

\bigskip

{\it Note:} Assume $\alpha \circ \Pi \circ \alpha^{-1}\begin{pmatrix}
0\\
0\\
0\end{pmatrix} = \begin{pmatrix}
0\\
0\\
0\end{pmatrix}$
\subsection*{Solution}
No. In order for $\alpha \circ \Pi \circ \alpha^{-1}$ to be linear, for any $\gamma,\,\delta \in (\mathbb Z_3 )^3$ it must be true that:
\[ \alpha \circ \Pi \circ \alpha^{-1}( \gamma + \delta) = \alpha \circ \Pi \circ \alpha^{-1}(\gamma) + \alpha \circ \Pi \circ \alpha^{-1}(\delta). \]
Consider the following:
\begin{align*}
\alpha \circ \Pi \circ \alpha^{-1}(A +A) &= \alpha \circ \Pi \circ \alpha^{-1}( \langle 0 0 1 \rangle + \langle 0 0 1 \rangle) \\
&= \alpha \circ \Pi \circ \alpha^{-1}( \langle 0 0 2 \rangle) \\
&= \alpha \circ \Pi \circ \alpha^{-1}(B) \\
& = Y \\ 
&= \langle 2 2 1 \rangle
\end{align*}
Additionally,
\begin{align*}
\alpha \circ \Pi \circ \alpha^{-1}(A) +\alpha \circ \Pi \circ \alpha^{-1}(A) &= Z + Z \\
&= \langle 2 2 2 \rangle + \langle 2 2 2 \rangle \\
&= \langle 1 1 1 \rangle \\
&= M
\end{align*}
Hence, $\alpha \circ \Pi \circ \alpha^{-1}(A + A) \not = \alpha \circ \Pi \circ \alpha^{-1}(A) + \alpha \circ \Pi \circ \alpha^{-1}(A)$.

\end{enumerate}

\item Let $T_i$ denote the $i$th triangular number, $i \ge 1$.
\begin{enumerate}\setlength{\itemsep}{6pt}

\item Derive formulas for the following:
\begin{enumerate}\setlength{\itemsep}{6pt}

\item $ \sum_{i=1}^n T_i $
\subsection*{Solution}
\begin{align*}
\sum_{i=1}^n T_i &= \sum_{i=1}^n \frac{i \cdot (i+1)} 2 \\
&= \frac 1 6 n (n+1) (n+2)
\end{align*}

\item $ \sum_{i=1}^n T_{2i-1} $
\subsection*{Solution}
\begin{align*}
\sum_{i=1}^n T_{2i-1} &= \sum_{i=1}^n \frac{i \cdot (4i-2)} 2 \\
&= \frac 1 6 n (n+1) (4n-1)
\end{align*}

\item $ \sum_{i=1}^{2n} (-1)^i T_i $
\subsection*{Solution}
\begin{align*}
\sum_{i=1}^{2n} (-1)^i T_i  &= \sum_{i=1}^n (-1)^i \cdot \frac {i\cdot(i+1)} 2 \\
&= \frac 1 8 (-1)^n \cdot [2n(n+2) + 1] -1
\end{align*}

\item $ \sum_{i=1}^n (T_i)^2 $
\subsection*{Solution}
\begin{align*}
\sum_{i=1}^n (T_i)^2  &= \sum_{i=1}^n \left(\frac{i\cdot(i+1)} 2 \right)^2 \\
&= \frac 1 {60} n (n+1) (n+2) (3n^2 + 6n +1)
\end{align*}

\item $ \sum_{i=1}^n (\frac 1 {T_i}) $
\subsection*{Solution}
\begin{align*}
\sum_{i=1}^n \left(\frac 1 {T_i} \right) &= \sum_{i=1}^n \frac{2}{i\cdot(i+1)} \\
 &= \frac{2n}{n+1}
\end{align*}
\end{enumerate}
\item
\begin{enumerate}\setlength{\itemsep}{6pt}

\item Under what conditions will $T_i$ be odd?
\subsection*{Solution}
$T_i$ will be odd if $i \mod 4 \equiv 1 \text{ or } 2$.

\item Prove that when represented in base ten, no triangular number will end in 4 or 7.
\begin{proof}
Recall that every $T_i$ is of the form $\tfrac {n(n+1)} 2$. We will proceed by contradiction. Suppose to the contrary that $\frac {n(n+1)} 2 \mod 10 \equiv 4$ and $\frac {n(n+1)} 2 \mod 10 \equiv 7$. First:
\[ \frac {n(n+1)} 2 \mod 10 \equiv 4. \]
Then we have the following:
\begin{align*}
\frac {n(n+1)} 2 \mod 10 &\equiv 4 \\
n(n+1) \mod 10 &\equiv 8 \\
n\mod 10 \cdot (n+1) \mod 10 &\equiv 8 \\
n\mod 10 \cdot n \mod 10 +1 \mod 10 &\equiv 8 \\
n\mod 10 \cdot n \mod 10 &\equiv 7 \\
n^2 \mod 10&\equiv 7
\end{align*}
I claim we have reached a contradiction, since $n^2$ cannot end in 7. Next, suppose the following:
\[ \frac {n(n+1)} 2 \mod 10 \equiv 7. \]
Similarly, we have:
\begin{align*}
\frac {n(n+1)} 2 \mod 10 &\equiv 7 \\
(n^2 + 1) \mod 10 &\equiv 4 \\
n^2 \mod 10 + 1 \mod 10 &\equiv 4 \\
n^2 \mod 10 &\equiv 3
\end{align*}
Again, I claim we've reached a contradiction, because $n^2$ cannot end in 3.
\end{proof}

\end{enumerate}

\item Prove that every {\it even} perfect number is a triangular number.
\begin{proof}
Let $p$ be a perfect number. Then $p = 2^{k-1} \cdot (2^k - 1)$. Let $n=2^k-1$, then:
\[ 2^{k-1} = \frac {2^k} 2 = \frac {n+1} 2. \]
Then we have $p = \frac {n+1} 2 \cdot n = \frac {n\cdot(n+1)} 2$. Since every $T_i$ is of the form $\frac {n\cdot(n+1)} 2$ for some $n\in\mathbb N$, we have shown that $p$ is a triangular number.
\end{proof}

\end{enumerate}

\item In this problem {\it Primitive} Pythagorean Triples will be denoted $(a, b, c)$ where $a$ is even and $a^2 + b^2 = c^2$, and Pythagorean Triples will be denoted $(x,y,z)$ where $x<y$ and $x^2+y^2=z^2$.
\begin{enumerate}\setlength{\itemsep}{6pt}
\item
\begin{enumerate}\setlength{\itemsep}{6pt}
\item Prove that there is {\it no} Primitive Pythagorean Triple having
\[c \equiv 3\mod 4\]
\begin{proof}
I claim that every primitive pythagorean triple $(a, b, c)$ is of the form:
\[ (a, b, c) = \left( \frac{m^2 - n^2} 2, mn, \frac{m^2 + n^2} 2 \right) \]
for some pair of relatively prime {\it odd} numbers $1 \le n < m$. 

Given $a^2 + b^2 = c^2$,  we can rewrite this equation as $b^2 = c^2 - a^2$ and factor the right side. Now we have:
\begin{align*}
b^2 &= (c+a) \cdot (c - a) \\
1 &= \left( \frac c b + \frac a b \right) \cdot \left( \frac c b - \frac a b \right)
\end{align*}
Since we have 1 on the left, the two terms on the right must be reciprocals of one another.

Let $\frac m n = \frac c b + \frac a b$ and $\frac n m = \frac c b - \frac a b$. Now we have:
\[ \frac c b = \frac 1 2 \cdot \left( \frac m n + \frac n m \right)
= \frac {m^2 + n^2}{2mn} \]
\[ \textrm{and} \]
\[ \frac a b = \frac 1 2 \cdot \left( \frac m n - \frac n m \right)
= \frac {m^2 - n^2}{2mn} \]

From this we can find that $a = \frac {m^2 - n^2} 2$, $b = mn$, and $c = \frac {m^2 + n^2} 2$.

\bigskip

Now we will show that there is no Primitive Pythagorean Triple having $c \equiv 3\mod 4$. Suppose on the contrary that:
\[ c \mod 4 = \frac {m^2 + n^2} 2 \mod 4 \equiv 3. \]
Then:
\begin{align*}
\frac {m^2 + n^2} 2 \mod 4 &\equiv 3 \\
m^2 + n^2 \mod 4 &\equiv 2,
\end{align*}
which implies that there exists some $i,\,j\in\mathbb Z$ such that $m = 2i$ and $n=2j$. I claim that we have reached a contradiction, because $m,\,n$ must be odd.
\end{proof}

\item What is the Pythagorean Triple having smallest $z$ such that
\[ z \equiv 3 \mod 4 \]
\subsection*{Solution}
(9,\,12,\,15), since $15 \equiv 3 \mod 4$.

\end{enumerate}
\item
\begin{enumerate}\setlength{\itemsep}{6pt}
\item Prove that there is {\it no} Primitive Pythagorean Triple having
\[c \equiv 11\mod 20\]
\begin{proof}
We've shown in (4)(a)(i) that primitive pythagorean triples are of the form:
\[ (a, b, c) = \left( \frac{m^2 - n^2} 2, mn, \frac{m^2 + n^2} 2 \right) \]
Suppose on the contrary that:
\[ c \mod 20 = \frac {m^2 + n^2} 2 \mod 20 \equiv 11. \]
Then we have that:
\begin{align*}
\frac {m^2 + n^2} 2 \mod 20 &\equiv 11 \\
m^2 + n^2 \mod 20 &\equiv 2,
\end{align*}
which implies that there exists some $i,\,j\in\mathbb Z$ such that $m = 2i$ and $n=2j$. I claim that we have reached a contradiction, because $m,\,n$ must be odd.
\end{proof}

\item What is the Pythagorean Triple having smallest $z$-value such that
\[z \equiv 11\mod 20\]
\subsection*{Solution}
(24,\,45,\,51), since $51 \equiv 11 \mod 20$.

\end{enumerate}
\item We say that a Primitive Pythagorean Triple is {\it simple} when $a=c-1$. Noting that $(4,3,5)$ is the {\it first} simple triple, that is, 5 is the {\it smallest} $c$-value, what is the {\it simple} triple that has the 100th smallest $c$-value?
\subsection*{Solution}
(20200,\,201,\,20201).
%We've shown that primitive pythagorean triples are of the form:
%\[ (a,b,c) = \left(\frac{x^2-y^2} 2, xy, \frac{x^2+y^2} 2\right) \]

\item As pointed out in class Primitive Pythagorean Triples may be listed according to the values of $t$ and $s$ where $t>s$, $\gcd{(t,s)} = 1$, and $s \not\equiv t \mod 2$.

As examples, the first 4 primitive triples have $(t,s) = (2,1), (3,2), (4,1)$, and (4,3) respectively. $(a,b,c) = (4,3,5), (12,5,13), (8,15,17)$, and (24, 7, 25).
\begin{enumerate}\setlength{\itemsep}{6pt}
\item In what position on the list will the 10th {\it simple} triple appear?
\subsection*{Solution}
The tenth simple triple is the 27th primitive pythagorean triple.

\item If $L_i$ denotes the position of the $i$th simple triple on the list, explain why $L_{2i+1}-L_{2i}$ is {\it even}, $i \ge 1$.

For example, when $i=1$, $L_3-L_2 = 4-2$.

Note also that $L_5-L_4 = 8-6$.

\end{enumerate}

\item Let $q$ be a {\it prime number} such that $q\equiv 3\mod 4$. $(q = 3,7,11,19,\ldots)$

Prove that {\it no} primitive triple has $c\equiv 0 \mod q$.

{\it Hint:} Consider the {\it group} $U(q)$. Note that $|U(q)| = \phi(q) = q-1$. Use results from group theory.
\end{enumerate}

\item In this problem you will consider integer solutions to the equation $x^2+y^2=z^4$. We shall refer to a solution of this equation as a Trinity triple. Moreover, if $(x,y,z)$ is a Trinity triple such that $\gcd{(x,y,z)}=1$, it shall be called primitive.
\begin{enumerate}\setlength{\itemsep}{6pt}
\item Prove that if $(a,b,c)$ is a primitive Trinity triple, then $a$, $b$, or $c$ is divisible by 5.
\begin{proof}
Obviously $a,\,b\mod 5 = 0, 1, 2, 3, \text{or } 4$. Therefore, $a^2,\,b^2\mod 5 = 1 \text{ or } 4$. 
Hence, $c \equiv (1+1)\mod 5$, $c \equiv (1+4)\mod 5$, or $c \equiv (4+4)\mod 5$.

Additionally, $c\mod 5 = 0, 1, 2, 3, \text{or } 4$. Therefore, $c^4\mod 5 = 0 \text{ or } 1$. 

We now know that $c^4 \not\equiv (1+1) \mod 5$, since $2 \mod 5 \not = 0 \text{ or } 1$. Similarly, $c^4 \not\equiv (4+4) \mod 5$, since $3 \mod 5 \not = 0 \text{ or } 1$.

Therefore, the only solutions to $a^2+b^2=c^4$ have one of $a,\,b$ being equivalent to $1\mod 5$ and the other being equivalent to $4\mod 5$. In this situation, $c^4 \mod 5 \equiv 0$, which means that $c^4$ is divisible by 5.
\end{proof}

\item Let (a,b,c) be a primitive Trinity triple of $a^2+b^2=c^4$.
\begin{enumerate}\setlength{\itemsep}{6pt}
\item Prove that if $b \equiv 3$ or $7\mod 10$, then $c$ is divisible by 5 and $a\equiv 4$ or $6 \mod 10$.
\item Give an example of a primitive Trinity triple that meets the conditions in (b)(i).
\end{enumerate}
\item Observing that if $(a,b,c)$ is a primitive Trinity triple, then $(a,b,c^2)$ is a primitive Pythagorean triple, use the results of primitive Pythagorean triples (values of $s$ and $t$) to demonstrate a one-to-one correspondence between the set of primitive Pythagorean triples and the set of primitive Trinity triples.
\end{enumerate}

\end{enumerate}

\end{document}
